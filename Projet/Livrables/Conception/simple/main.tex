\documentclass[12pt]{article}
\usepackage[french]{babel}
\usepackage[T1]{fontenc}
\usepackage[utf8]{inputenc}
\usepackage{xspace}
\title{Dossier de conception}
\shorttitle{G \& L : Conception}
\authors{H4215}
\date{}
\newcommand{\kw}[1]{\texttt{#1}}
\newcommand{\TODO}[1]{{\huge TODO : #1}}
\begin{document}
\maketitle
\tableofcontents
\newpage
\section{Modélisation d'un document XML}
\TODO{Diagramme}
La structure d'un document XML est modélisée de manière relativement classique. On notera l'utilisation d'un \emph{design pattern} \og composite \fg.
Le

\section{Modélisation d'une DTD}
\subsection{Structures de données}
\TODO{Diagramme}
\TODO{Classes DTD, Element, etc.}
\kw{Content} représente le contenu d'un élément dans la DTD, c'est-à-dire la dernière partie de la balise \kw{<!ELEMENT nom contenu>}. Il permet d'effectuer une validation de la structure d'un noeud XML. Il définit pour cela une interface simple composée des deux versions de la méthode \kw{validate} ; celle-ci renvoie vrai ou faux en fonction de l'adéquation du contenu du noeud passé en paramètre avec le contenu (classe \kw{Content}) sur lequel a été appelé la méthode. On suppose, lorsque ces méthodes sont appelées, que la version appelée correspond au type réel du noeud.
Parmi les classes dérivées directement de \kw{Content}, on retrouve :
\begin{description}
\item[\kw{EmptyContent}] Contenu vide. Valide n'importe quel noeud de type \kw{MarkupNode}.
\item[\kw{AnyContent}] Contenu quelconque. Valide n'importe quel noeud de type \kw{MarkupNode} ou \kw{CompositeMarkupNode}.
\item[\kw{BrowsableContent}] Contenu plus complexe, potentiellement composite et organisé sous forme d'arbre, et donc navigable pour la validation.
\end{description}
\subsection{Validation}
La validation consiste, à partir de la liste des noeuds fils d'un noeud XML, à leur faire correspondre un arbre de contenu (classe \kw{Content}).
\subsubsection{Algorithme général}
L'algorithme de validation effectue un parcours en profondeur de l'arbre de contenu, avec backtracking. Dans la suite du document, nous étudierons séparément la navigation dans l'arbre, le backtracking et la validation de chaque type particulier de contenu.
\subsubsection{Navigation}
Une validation suit, classiquement, le parcours suivant dans l'arbre :
\begin{figure}
\caption{\TODO{Image parcours en profondeur}}
\end{figure}
La classe abstraite \kw{BrowsableContent} met en place l'ensemble des méthodes nécessaires à la navigation dans l'arbre de contenu.
En voici une description succinte :
\begin{description}
\item[\kw{+validate(:CompositeMarkupNode):boolean}] Effectue la validation proprement dite (interface client).
\item[\kw{\#newValidation(firstToken:ChildrenIterator, endToken:ChildrenIterator, nextStep:BrowsableContent):boolean}] Méthode protégée appelée sur le contenu suivant lors d'un accès en \og descendant \fg dans l'arbre. Tente de valider (de \og matcher \fg) un maximum de jetons (noeuds enfants) de la liste donnée en paramètre, avec pour contrainte que la validation de la prochaine étape reste d'actualité.
\item[\kw{\#continueValidation(currentToken:ChildrenIterator):boolean}] Méthode protégée appelée sur le contenu suivant lors d'un accès en \og remontant \fg dans l'arbre. Continue la validation entreprise lors du dernier appel à newValidation sur le même objet.
\end{description}
Ainsi, la navigation pourrait être modélisée comme suit :
\begin{figure}
\caption{\TODO{Image parcours en profondeur newValidation & continueValidation}}
\end{figure}
Cependant, les méthodes \kw{newValidation} et \kw{continueValidation} sont protégées, et ne peuvent donc pas être appelées depuis d'autres instances de classes dérivées (en C++, tout au moins). Afin d'accéder à ces méthodes (protégées) depuis d'autres objets de classes dérivées de \kw{BrowsableContent}, il faut définir les méthodes suivantes :
\begin{description}
\item[\kw{\#browseDown(childContent:BrowsableContent, firstToken:ChildrenIterator, endToken:ChildrenIterator, nextStep:BrowsableContent):boolean}] Exécute la descente dans l'arbre de contenu. Consiste à appeler la méthode \kw{newValidation} sur le contenu \kw{childContent}.
\item[\kw{\#browseUp(parentContent:BrowsableContent, currentToken:ChildrenIterator, endToken:ChildrenIterator):boolean}] Exécute la remontée dans l'arbre de contenu. Consiste à appeler la méthode \kw{newValidation} sur le contenu \kw{childContent}.
\end{description}
Ces méthodes sont appelées sur lui-même par un objet désirant continuer la navigation.
On obtient le schéma de navigation suivant :
\begin{figure}
\caption{\TODO{Image parcours en profondeur newValidation & continueValidation}}
\end{figure}
En réalité, la méthode \kw{newValidation} n'implémente pas vraiment la validation. Elle se contente d'appeler des patrons de méthodes permettant d'effectuer un pré-traitement, une validation proprement dite, et un post-traitement. Cela permet d'effectuer des empilements d'états nécessaires à l'implémentation des opérateurs de quantification (cf. \TODO{}). Les patrons en questions (méthodes virtuelles définies dans les classes dérivées) sont les suivants :
\begin{description}
\item[\kw{\#beforeValidation(firstToken:ChildrenIterator, endToken:ChildrenIterator, nextStep:BrowsableContent):boolean}] Pré-traitement précédant une validation.
\item[\kw{\#startValidation(firstToken:ChildrenIterator, endToken:ChildrenIterator, nextStep:BrowsableContent):boolean}] Première étape d'une validation, potentiellement suivie d'appels à \kw{continueValidation}.
\item[\kw{\#afterValidation():boolean}] Post-traitement précédant une validation.
\end{description}
Pour résumer, voici un diagramme de séquences représentant le schéma type d'une navigation :
\begin{figure}
\caption{\TODO{Diagramme de séquences BrowsableContent::validate}}
\end{figure}
\subsubsection{Backtracking}
Etant donnée la manière dont l'arbre est parcouru, chaque noeud effectuant un choix (\kw{Choice}, \kw{QuantifiedContent}, \kw{MixedContent}) est mis au courant si la suite de la validation n'a pas pu se faire. Il est donc possible, au niveau de chacun de ces noeuds, de gérer le backtracking :
\begin{itemize}
\item après un \kw{Choice} ou un \kw{MixedContent}, si la suite de la validation n'a pas pu être effectuée, on passe au choix suivant. Si on a atteint la fin de la liste, on renvoie faux pour signaler au contenu parent qu'il faut effectuer un backtracking.
\item après un \kw{QuantifiedContent}, si la suite de la validation n'a pas pu être effectuée, on tente de \og matcher \fg un jeton de moins, puis de relancer la suite de la validation. Si on arrive en dessous du nombre minimum de \og matching \fg, on renvoie faux pour signaler au contenu parent qu'il faut effectuer un backtracking.
\end{itemize}
\subsubsection{Validation spécifique}
\TODO{Description spécifique des matchings}
\newpage
\section{Algorithme}
\subsection{Validation}
Parcourir arbre pour obtenir liste avec tous les noeuds.
Pour chaque noeud texte
**Vérifier qu’il s’agit d’un mixed contexte
Pour chaque noeud markup
\end{document}

