\section{Algorithme transformation XSL}
\begin{algorithm}[H]
\caption{\kw{Transformation(docXml:document, docXSL:document) : Noeud}}
\begin{algorithmic}
\STATE \COMMENT{ Méthode qui prend en paramètre les fichiers XML et XSL de départ et renvoie le fichier XML final }
\STATE    xml = analyser(docXml) /COMMENT{appel la méthode pour la construction de la structure XML}
\STATE   xsl = analyser(docXSL) /COMMENT{idem}

 \STATE \COMMENT{pour pouvoir accéder rapidement au template}
   \STATE global_map = create Map<paire<NamespaceTemplate,NomTemplate>, NoeudTemplate>();    

     \STATE  xmlTransform = analyse_noeud(vide, xml.racine());

       \IF{ (xmlTransform.size() > 1)}
        \STATE \COMMENT{ le template appliqué à la racine a donné plusieurs arbre, ce qui est incorrect }
        \RETURN vide
    \ELSE
        \RETURN xmlTransform[0]
    \ENDIF
\END
\end{algorithmic}
\end{algorithm}


\begin{algorithm}[H]
\caption{\kw {rechercher_template(x:Noeud):NoeudComposite}}
\begin{algorithmic}
\STATE \COMMENT{Renvoie le template correspondant à x, ou vide s’il n’y en a pas }

    \IF{x is TextNode}
    \STATE \COMMENT{ Un textNode ne peut pas correspondre à un template}
        \RETURN vide
    \ELSE \STATE \COMMENT{x is MarkupNode ou x is CompositeMarkupNode}
        \RETURN global_map.find(x.namespace,x.name)
    \ENDIF
end
\end{algorithmic}
\end{algorithm}


\begin{algorithm}[H]
\caption{\kw{analyser_noeud(parent:Noeud, x:Noeud):liste<Noeuds>}}
\begin{algorithmic}
\STATE \COMMENT{ Applique la transformation XSL à un noeud en fonction de ses caractéristiques }
    \STATE Template tplt = rechercher_template(x); /COMMENT{recherche un template qui matche x}

    \IF {tplt == vide} \STATE \COMMENT{si aucun template n’est adapté}
        \STATE \COMMENT{On renvoie directement une copie du noeud}
           \RETURN copie_simple(parent,x);
    \ELSE
        \STATE \COMMENT{ Un template a été trouvé, il faut renvoyer la copie des fils de ce template
        prennant en compte, le cas échéant, le remplacement des apply-templates
        par les fils de x }
           var remplaçants:liste<Noeuds>

      \STATE foreach (fils Ftplt de tplt)
      \STATE     remplaçants.concat(copie_xsl(parent,Ftplt,x))

     \STATE  endforeach

           \RETURN remplaçants
       \ENDIF
\END
\end{algorithmic}
\end{algorithm}


\begin{algorithm}[H]
\caption{\kw{copie_xsl(parent,tplt,x):liste<Noeud>}}
\begin{algorithmic}
\STATE \COMMENT{Copie un arbre XSL, et y intègre les fils d’un noeud XML à la place des éventuels noeuds     <xsl:apply-templates/>}
    \STATE var resultat:liste<Noeud>

    \IF{tplt is TextNode}
       \STATE    resultat.append(copie_simple(parent,t))
       \ELSE
	   \IF {tplt is CompositeMarkupNode}
           \STATE resultat.append(CompositeMarkupNode(parent,tplt.ns,tplt.name, copie_xsl des fils de tplt)}
    \ELSEIF (tplt is MarkupNode)
           \IF (tplt is xsl:apply-templates)
            \IF (x is CompositeMarkupNode)

              \STATE foreach(fils Fx de x)
                    \STATE   resultat.concat(analyser_noeud(parent,Fx))
              \STATE   endforeach
            \ELSE
                \STATE \COMMENT{ ne rien faire }
            \ENDIF

       \ELSE
            \STATE   resultat.append(copie_simple(parent,t))

       \ENDIF
       \ENDIF

    \RETURN resultat
\END
\end{algorithmic}
\end{algorithm}



\begin{algorithm}[H]
\caption{\kw{copie_simple(parent,x) : Noeud}}
\begin{algorithmic}
\IF {x is TextNode}
       \RETURN TextNode(parent,x.texte)
   \ELSE
   \IF {x is CompositeMarkupNode}
       \RETURN CompositeMarkupNode(parent,x.ns,x.name, analyse_noeud des fils de x)
   \ELSE
   \IF {x is MarkupNode}
       \RETURN MarkupNode(parent,x.ns,x.name)
   \ENDIF
\END

\end{algorithmic}
\end{algorithm}

\end{document}
